\chapter{Revis�o Bibliogr�fica}

Neste cap�tulo, s�o apresentados alguns fundamentos te�ricos necess�rios para o
entendimento desta disserta��o e alguns dos principais trabalhos e pesquisas
cient�ficas relacionados ao controle de miss�o de rob�s m�veis. O objetivo deste
levantamento bibliogr�fico � apresentar t�cnicas vi�veis e aplica��es de
diversos sistemas de controle de miss�o, e direcionar o leitor para os
conhecimentos aplicados no controle de miss�o do rob� DORIS, projeto que
envolve esta disserta��o.
t�cnicas em diversos principais pesquisas No in�cio dos anos 50, LANCZOS~\cite{Lan50a} desenvolveu um m�todo para a solu��o de problemas de autovalor sim�tricos, empregando transforma��es similares, obtendo uma matriz da qual fosse mais simples obter seus autovalores do que a matriz original do problema. ARNOLDI \cite{Arn51a} generalizou o m�todo, estendendo-o para problemas n�o sim�tricos.

